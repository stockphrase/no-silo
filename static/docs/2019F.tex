\documentclass[letterpaper]{article}

\raggedbottom

\usepackage[left=.8in, top=.8in, right=.8in, bottom=.8in]{geometry}

\usepackage{fontspec}

\usepackage{cancel}

%%% Quotations  %%%%%%%%%%%%

\usepackage{csquotes}
\MakeOuterQuote{"}


\setlength{\parindent}{0pt}
\usepackage[parfill]{parskip}

%%%%%%%%%%%%%%%%%%%%%%%%%%%

%% Use black squares for bullets

% \usepackage{amssymb}
% \renewcommand{\labelitemi}{\tiny$\blacksquare$}

%%%%%%%%%%%%%%%%

\usepackage{color, graphicx, wrapfig}
\usepackage{hyperref}
\hypersetup{
  colorlinks,
  linkcolor=black,
  urlcolor=Ahrenge}

\usepackage{soul}
\usepackage{pifont}

\usepackage[table]{xcolor} 



%%%%%%%%%% TColorBox %%%%%%%%%%%%

\usepackage[many]{tcolorbox}
%----------------
\newtcbox{\mylib}{enhanced,nobeforeafter,tcbox raise base,boxrule=0.4pt,top=0mm,bottom=0mm,
  right=0mm,left=4mm,arc=1pt,boxsep=3pt,before upper={\vphantom{dlg}},
  colframe=Ahrenge!80!black,coltext=Ahrenge!25!black,colback=Ahrenge!10!white,
  overlay={\begin{tcbclipinterior}\fill[Ahrenge!75] (frame.south west)
    rectangle node[text=white,font=\sffamily\bfseries\small] {\ding{72}} ([xshift=4mm]frame.north west);\end{tcbclipinterior}}}

\robustify{\mylib}

\pdfstringdefDisableCommands{%
  \def\mylib#1{'#1'}%
}
%----------------



%----------------
\newtcbox{\mylibbook}{enhanced,nobeforeafter,tcbox raise base,boxrule=0.4pt,top=0mm,bottom=0mm,
  right=0mm,left=4mm,arc=1pt,boxsep=3pt,before upper={\vphantom{dlg}},
  colframe=Ahrenge!80!black,coltext=Ahrenge!25!black,colback=Ahrenge!10!white,
  overlay={\begin{tcbclipinterior}\fill[Ahrenge!75] (frame.south west)
    rectangle node[text=white,font=\sffamily\bfseries\tiny, rotate=90] {GET} ([xshift=4mm]frame.north west);\end{tcbclipinterior}}}

\robustify{\mylibbook}

\pdfstringdefDisableCommands{%
  \def\mylibbook#1{'#1'}%
}
%%%%%%%%%% /end TColorBox %%%%%%%%%%%%




%%%%%%%%%% CheckBoxes %%%%%%%%%%%%
\usepackage{enumitem,amssymb}
\newlist{todolist}{itemize}{2}
\setlist[todolist]{label=$\square$}
%%%%%%%%%% CheckBoxes %%%%%%%%%%%%

%%%%%%%%%% Watermark %%%%%%%%%%%%

%\usepackage{draftwatermark}
%\SetWatermarkText{Draft. Do Not Distribute.}
%\SetWatermarkScale{.5}
%%%%%%%%%% Watermark %%%%%%%%%%%%

\usepackage{pstricks}
\usepackage[framemethod=tikz]{mdframed}


\usepackage{pgflibraryarrows}


\usepackage{dashrule}

%%%%%%% Wallpaper

\usepackage{wallpaper}

%%%%%  Definitions %%%%%%%%%%%%%%%%

\def\mybibliostyle{plain}
\def\mybibliocommand{}
\def\mysubtitle{}
\def\myaddress{37 Dewey Field Road, Room 224 }
\def\myemail{\href{mailto:alan.c.taylor@dartmouth.edu}{alan.c.taylor@dartmouth.edu}}
\def\myweb{www.opensourcewriting.org}
\def\myphone{603.646.1241}
\def\myaffiliation{\ \\Dartmouth College}
\def\mykeywords{Writing, Rhetoric, Undergraduate, Education}
\def\myofficehours{Tuesday 12-2 (and by appointment) }
\def\canvaslogin{https://canvas.dartmouth.edu}

\def\mytime{11 }
\def\mysection{.06 }
\def\myauthor{Alan C. Taylor}
\def\mytitle{What is Education?}
\def\mysubtitle{Syllabus}
\def\mycopyright{\myauthor}
\def\mykeywords{}


\def\myta{\href{mailto:julia.s.ceraolo.gr@dartmouth.edu}{julia ceraolo}}
%\def\myta{\href{mailto:janina.l.misiewicz.gr@dartmouth.edu}{janina misiewicz}}


%============================================


% COLORS

\definecolor{SteelGray}{RGB}{88, 89, 91}
\definecolor{Ahrenge}{RGB}{220, 109, 27}
%\definecolor{Ahrenge}{RGB}{255, 105, 1}
\definecolor{oyster}{RGB}{222, 222, 222}

% Highlight Color

\DeclareRobustCommand{\hloy}[1]{{\sethlcolor{oyster}\hl{#1}}}


%% Use squiggles for bullets

\usepackage{amssymb}
\renewcommand{\labelitemi}{$\rightsquigarrow$}

%%%%%%%%%%%%%%%%%%%%%%%%55

% FONTS

\setromanfont{Alegreya}
\setsansfont{Alegreya}
\setmonofont{Alegreya}

\newfontfamily\specialfont{Economica}
\newfontfamily\portfolio{Ubuntu}
\newfontfamily\sectionfont{Alegreya SC}


\usepackage{fancyhdr}
\fancyfoot[C]{\{ \thepage \}}

\title{What is Education?}
\date{}

% Stylin'

\usepackage{titlesec}
\titleformat{\section}
{\color{black!70}\sectionfont\Large\bfseries}
{\color{SteelGray}\thesection}{1em}{}


\begin{document}

{\specialfont \ding{96}  fall \the\year\ | writing 002\mysection | \mytime | ta: \myta}
\vspace{1cm}

{\Huge \mytitle}

\textcolor{SteelGray!60}{\hrule}

\textcolor{Ahrenge!60}{\hrule}


\bigskip

\begin{tabular}{ll}
{\Large Alan C. Taylor} & \\
& \\
Office: & {\href{https://goo.gl/maps/5ACU6kGcqv7JcAzf7}{\myaddress}} \\
Hours: & \myofficehours \\
Email: & \myemail \\
%Web: & \href{\myweb}{\myweb} \\
Canvas: & \href{https://canvas.dartmouth.edu}{canvas.dartmouth.edu} \\

\end{tabular}

\section*{Course Description}

Writing 2-3 is an introduction to academic writing and research. The overarching objective of this course is to prepare you for the rigors of the academic life. To succeed in a scholarly environment we must develop good habits of inquiry, analysis, and communication. We will cultivate these critical practices through rigorous group inquiry\textemdash by reading and analyzing a series of demanding texts and articulating responses to them in the form of essays. Shorter writing assignments will focus on a number of skills that are important for taking part in academic conversations. Among these are the formal documentation of sources and the integration of source materials through summary, paraphrase, and quotation. 

During the Winter term we will turn our focus to academic research. Our libraries have an impressive array of traditional and electronic search tools as well as millions of books, journal articles, and assorted media. Although navigating this vast sea of information is intimidating, it is important that you find your way: excellent research skills are fundamental to your undergraduate training, regardless of your chosen field of study. In consideration of its importance, we will spend a significant amount of time learning how to use our library effectively. 

A more detailed description of the course objectives and outcomes may be found in the \href{http://dartmouth.edu/writing-speech/curriculum/writing-courses/writing-2-3/writing-2-3-course-outcomes}{Writing 2/3 Course Outcomes} statement on the Institute for Writing and Rhetoric website.


 
\section*{Required Texts}

\href{https://github.com/stockphrase/OpenHandbook/raw/master/Open%20Handbook.pdf}{\mylibbook{\emph{The Open Handbook}}} 

\ding{96} Other course readings may be downloaded from \href{https://canvas.dartmouth.edu}{Canvas}.



\section*{Academic Honesty}
All work submitted in this course must be your own and be written exclusively for this course. The use of sources (ideas, quotations, paraphrase) must be properly documented. Please read the \href{https://students.dartmouth.edu/judicial-affairs/policy/academic-honor-principle}{Academic Honor Principle} for more information about the dire consequences of plagiarism. If you are confused about when or how to cite information, please consult the  course handbook or ask me about it before submitting your work. %Additionally, the \href{http://writing-speech.dartmouth.edu/learning/materials/sources-and-citations-dartmouth}{Sources and Citations} document on the Institute for Writing and Rhetoric website contains valuable information on documentation and avoiding plagiarism.

\section*{Attendance}
Regular attendance is expected. Bracketing religious observance, severe illness, or personal tragedy, no more than three unexcused absences in a single term will be acceptable for this course. \emph{This policy applies to regular class meetings, assigned X hours, and TA meetings.} Four or more unexcused absences may result in repercussions ranging from significant reduction in GPA to failure of the course. If you have a religious observance that conflicts with your participation in the course, please meet with me beforehand to discuss appropriate accommodations. 

\section*{Papers}
You will submit three formal essays and several smaller writing assignments in response to our readings and class discussions. The formal essays focus on argumentation, synthesis, close reading, and theoretical analysis. The exact nature and objective of each assignment will be explained in greater detail throughout the course of the term. 

\begin{itemize}

\item \textbf{Formatting}: During the Fall term papers must be submitted in the MLA format; in the Winter term we will transition to the Chicago style. Information on these styles may be found in your handbooks.
\item \textbf{Late Papers}: Late papers will receive \emph{a full letter grade deduction} for each day late.

\item \textbf{Revisions}: It is my practice to return essay drafts to you within one week. Afterward, you will have a minimum of one week to revise your writing. During this two-week revision cycle, I am happy to look at your drafts at any stage of development. Please feel free to meet with me during office hours (or by appointment) to go over your writing or discuss your ideas.
\end{itemize}


\section*{Reading, Thinking, Discussing, Reflecting, Writing}

Each week of this course follows a relatively predictable pattern. Mondays and Wednesdays are discussion days, where we commonly examine a reading together in class; Fridays are workshop days, where we learn and practice certain critical skills that are used in academic writing and conversations. Here is a list of the typical assignments and activities that we will do each week: 

\begin{itemize}

\item Before we meet to discuss a reading as a class, each of you will carefully read and critically engage the text on your own\textemdash interrogating, analyzing, and questioning the arguments, ideas, and assumptions you discover there. 

\item As you read the text, I ask that you \textbf{annotate} it\textemdash that is, mark up the text by adding meaningful symbols, marginal notes, and questions on the document itself. Bring this annotated copy of the reading with you to class to help you engage in the group discussion and analysis. 

\item After annotating the text, take \textbf{critical notes} on it. These notes will be invaluable to you later, when you write your essays. 

\item After annotating and taking notes, post a \textbf{critical response} to the text on the discussion forum in \href{https://canvas.dartmouth.edu}{Canvas}. Attach your critical notes to the response post you submit in Canvas.

\item Finally, write a \textbf{reflective journal entry} each week that catalogs your thoughts and ideas for the week.

\end{itemize}

The following four numbered sections explain each of these assignments in more detail:   

\subsection*{1 -- Annotating Texts}

Rather than use a laptop or tablet to read our course readings, \emph{I ask that you print them out and annotate them as part of your preparations for class}. Annotation refers to the process of marking up a text by adding your own words and symbols to the document itself. There is no right or wrong way to mark up a text, but you should develop a system that you are comfortable with and try to stick with it. Your objective in annotation is to flag the key elements of a piece of writing\textemdash the thesis, argumentative points, and pieces of evidence. In addition, use the margins of the text to ask questions, make brief notes, indicate confusion, define unfamiliar terms, and make connections to other texts. This kind of work serves two purposes: first, it helps you maintain a critical focus as you read; second, it helps you later if the text must be used for study or your own writing. If you plan on being successful in college, the ability to rigorously annotate texts is perhaps the most helpful and important skill you can develop. Further advice and caveats about annotation may be found in the \href{https://github.com/stockphrase/OpenHandbook/raw/master/Open%20Handbook.pdf}{\emph{Open Handbook}}. 

\subsection*{2 -- Critical Reading Notes }
Create an electronic document for composing critical notes. Take \emph{detailed} notes on each reading. Since you will write essays about these texts, these notes will be of significant help to you later. Your aim here should be to \emph{reduce the entire argument to its bare essentials using paraphrase, summary, and selective quotation}. Carefully document page numbers during this activity. Interrogate the text by asking questions, raising objections, and making observations. Connect and compare the reading to others we have read. Link to any outside research you perform and define unfamiliar terms or words. At the end of this process you should have a simplified version of the essay as well as a number of critical observations, questions, and ideas that emerged in the process of reading. 

\begin{itemize}
\item For more detailed information on the creation and purpose of these notes, read the chapters entitled ``Annotation \& Critical Reading'' and "Critical Notes" in the \href{https://github.com/stockphrase/OpenHandbook/raw/master/Open%20Handbook.pdf}{\emph{Open Handbook}}. 


\item \mylib{Attach your critical notes file to your response post for each reading in the \href{https://canvas.dartmouth.edu}{Canvas} discussion forum.} 
\end{itemize}

\subsection*{3 -- Reading Response Posts}

Before we meet to discuss a new course text, write a critical response and post it to our online discussion forum in \href{https://canvas.dartmouth.edu}{Canvas}. The post should discuss ideas, arguments, or questions that arise from the reading(s). These are writing and thinking exercises, not polished essays; however, I do expect that your posts will grapple with the readings in an intelligent way and demonstrate a great deal of thought and care. Here are some principles to consider as you craft your posts:

\begin{itemize}
\item \emph{Good posts do not merely summarize}; rather, they should seek to \emph{evaluate} and \emph{engage} the writer's claims or ideas. Use your posts to try to come to grips with the reading's ideas (or some aspect of them).

\item \emph{Good posts may seek to take issue with some of the thinking put forth in the reading}. However, a good post doesn't just say "I agree with X" or "I disagree with Y." Instead, \emph{explicit reasons are stated} and \emph{explanations are made} that challenge or support the writer's ideas. As you will discover, part of this class involves learning to take responsibility for your claims and arguments. This can only be done by providing reasons for the views you hold.

\item \emph{Good posts might alternatively attempt to forge a connection with another reading by demonstrating a relationship between the ideas or arguments involved}. For example, how might Author A respond to Author B? How do their views compare? Can their views be reconciled? Be specific. Show evidence.

\item \emph{Good posts may ask questions or express confusion}. Ignorance or confusion is not something to be feared. As Henry David Thoreau \href{https://www.gutenberg.org/files/1022/1022-h/1022-h.htm}{once argued}, ignorance is often more useful than positive knowledge: confusion generates questions; questions inspire inquiry; inquiry produces knowledge and more interesting questions. Although you are free to pose questions in your posts, you should also make good efforts to answer them. Questions exist to drive us toward answers, not leave us spinning our wheels.

\item \emph{Good posts are not sloppy and demonstrate a high degree of care}.

\item \emph{Good posts avoid narcissism}. Don't make everything about you or something that happened to you. While you will of course draw on your own experiences, your focus should be on the text(s).

\item \emph{Good posts are not derivative}. By this I mean that your post should be original and not a parasite clinging to the work of other students. Demand that everyone contribute their unique and individual perspective to our discourse community. Posts that clearly derive their very lifeblood from other posts should be met with hostility by the class. One way to avoid this problem is to post your response before reading the work of others. This does not mean that you should avoid responding to the thoughts of your colleagues. You should feel free to disagree or question the ideas of others (including your professor). This behavior is actually healthy for our discourse community.

\item \mylib{Posts are due by 10pm \textbf{on the evening prior} to our first discussion of a text (generally on Sundays).}

\end{itemize}


\subsection*{4 -- End-of-Week Journal Reflection}

Create an electronic document for use as a journal. By Friday each week, compose a reflective post in that document that looks back over the week's reading and discussions. What is your big takeaway for this week? What was the most meaningful thing you heard, thought, or read? \emph{It is perfectly fine if your entry deals with things other than this course}. Write as much as you like, but submit at least a half a page. We'll often talk about these on Fridays and I may periodically ask to you to turn them in.

\section*{Typical Weekly Workflow and Checklist}

\begin{todolist}

\item Locate the new reading for the upcoming week in \href{https://canvas.dartmouth.edu}{Canvas}. Download it. Print it out.
\item Read it carefully. Slow down. Unplug. Take your time.
\item Annotate as you read: make notations, flag important details, ask questions.
\item Practice deep note-taking. Reduce the argument to its essentials. Add your own thoughts/observations/questions.
\item On the night before a new reading is due to be discussed, post a reading response to \href{https://canvas.dartmouth.edu}{Canvas} and attach your critical notes file to it. (This process is referred to as \href{https://canvas.dartmouth.edu/}{\textbf{Post \& Notes}} below).
\item Before class on Friday, write a reflective journal entry that takes stock of the week.

\end{todolist}

\section*{{\portfolio e}Portfolios}

A large \href{http://www.opensourcewriting.org/opensourcewriting/wp-content/uploads/ePortfolios.pdf}{body of academic research} indicates that students who curate their academic performances over time and periodically revisit them experience measurable benefits in learning and development. For this reason, the \href{https://writing-speech.dartmouth.edu/dartwrite}{DartWrite} program provides every incoming student with a WordPress blog for use as an ePortfolio. This portfolio is yours. You own it. You may design it in any way you choose. You may post anything you like there\textemdash your writing, photography, reflections, whatever. We will talk more throughout the term about these ePortfolios and how they may be used. To access your ePortfolio, visit the login page at \href{https://journeys.dartmouth.edu/}{journeys.dartmouth.edu}. Use your NetID credentials to log in. 

\section*{Computer Policy}

I ask that you \textbf{do not use laptops or tablets} in class to take notes or reference the course readings. If you require some kind of technological accommodation that runs counter to this policy, please discuss this with me as soon as possible. 

 
\section*{Help With Your Writing}
There are many sources of help for your writing assignments. I am happy to meet with you all term during my office hours or by appointment. Each of you will meet with your TA for 45 minutes per week to go over your writing and plan revision. If you require further help, the \href{https://students.dartmouth.edu/rwit/}{RWIT} program offers excellent peer tutoring on all phases of the writing process\textemdash from generating ideas to formal citation.

\section*{Students With Disabilities}

All students are entitled to full access to this course, regardless of disability. If you have a disability and anticipate needing accommodations in this course, please contact me as soon as possible to arrange a confidential meeting. Students requiring disability-related services must register with the \href{http://www.dartmouth.edu/~accessibility/}{Student Accessibility Services} office. Once SAS has authorized services, students must show the originally signed SAS Services and Consent Form and/or a letter on SAS letterhead to
me. As a first step, if you have questions about whether you qualify to receive academic
adjustments and services, you should contact the SAS office. All inquiries and discussions will
remain confidential.


\section*{Grading Breakdown}
%Writing 2/3 is a two-term course. You will not receive a final grade at the end of the Fall term. Rather, if you have a C- average or better at the conclusion of the Fall term, you will receive an ``ON'' (for ``ongoing''). After completing Writing 3 during the Winter term you will receive a final grade that will be retroactively applied to both Fall and Winter terms. 

The following assignments comprise your grade for the Fall term:

\begin{tabular}{ l p{10cm} }
  Essay One: & 20\%\\ 
  Essay Two: & 30\%\\
  Essay Three: & 30\% \\
  Workshops: & 10\%\\
  Online posts: & 10\%\\
\end{tabular}


\section*{Schedule of Readings and Assignments}

\begin{small}
\begin{tabular}{ | p{.6cm} | p{5cm} | p{6.5cm} | p{3.5cm}|}

\hline
\textbf{Date} & \textbf{Readings} & \textbf{\ding{72} Assignments and \ding{96} Discussion Questions} & \textbf{Due}\\
\hline

9.16 & & Introductions, course objectives, syllabus, questions.

 & \\ \hline

9.18 &\href{https://github.com/stockphrase/OpenHandbook/raw/master/Open%20Handbook.pdf}{\emph{Open Handbook}}: read ch. 1, ``Annotating and Critical Reading'' and ch. 2, "Critical Notes" & \ding{96} What is education? 

\smallskip \ding{72} \emph{Annotation Exercise}

 & \\ \hline

9.20 & \href{https://github.com/stockphrase/OpenHandbook/raw/master/Open%20Handbook.pdf}{\emph{Open Handbook}}: read ch. 3, ``The Joy of Reuse'' & \ding{72} Bring two hard copies of your essay to class.

\smallskip \ding{72} \emph{Peer Workshop}.& \ding{72} \textbf{Essay 1.0} \\ \hline


\end{tabular}

\end{small}

\begin{small}
\begin{tabular}{ | p{.6cm} | p{5cm} | p{6.5cm} | p{3.5cm}|}

\hline 
9.23 & Nicholson Baker, ``Changes of Mind.'' & 
  & \href{https://canvas.dartmouth.edu/}{\ding{72} \textbf{Post \& Notes}}\\
\hline
9.25 & Re-read Baker. &  & \\
\hline
9.27 & \href{https://github.com/stockphrase/OpenHandbook/raw/master/Open%20Handbook.pdf}{\emph{Open Handbook}}: read ch. 8, ``Working with Sources'' and ch. 10, ``Plagiarism.'' & \ding{72} \emph{Workshop 1: Summary, Signal Phrases}.  & \ding{72} Reflective Journal\\

\hline


\end{tabular}

\end{small}

\begin{small}
\begin{tabular}{ | p{.6cm} | p{5cm} | p{6.5cm} | p{3.5cm}|}
\hline

9.30 & Walker Percy, ``The Loss of the Creature.'' &
\medskip

& \href{https://canvas.dartmouth.edu/}{\ding{72} \textbf{Post \& Notes}}\\
\hline

10.2 & Re-read Percy.&  & \\ \hline

10.4 & \href{https://github.com/stockphrase/OpenHandbook/raw/master/Open%20Handbook.pdf}{\emph{Open Handbook}}: read ch. 9, ``Altering Sources''; \emph{review} Ch. 12, ``MLA Style.'' 
\smallskip

 & \ding{72} \emph{Workshop 2: Acknowledging sources: documentation, summary, paraphrase, quotation}.
\smallskip

\ding{72} Bring two hard copies of your summary to class. & \ding{72} \textbf{Workshop 1} \smallskip 

\ding{72} Reflective Journal \\ \hline



\end{tabular}
\end{small}



\begin{small}
\begin{tabular}{ | p{.6cm} | p{5cm} | p{6.5cm} | p{3.5cm}|}
\hline

10.7& Paulo Freire, “The Banking Concept of Education.”&
&
\href{https://canvas.dartmouth.edu/}{\ding{72}\textbf{Post \& Notes}}\\ \hline

10.9&

Re-read Freire.& & \\ \hline

10.11& \href{https://github.com/stockphrase/OpenHandbook/raw/master/Open%20Handbook.pdf}{\emph{Open Handbook}}: read the ``Synthesis'' section in ch. 5, ``Types of College Writing'' & \ding{72} \emph{Workshop 3: Synthesis}.
\smallskip

\ding{96} What concerns, ideas, arguments, or terms do Percy and Freire share? How do they compare? Are they concerned about the same problem(s)? Are their conclusions/solutions similar or different? 
& \ding{72} \textbf{Workshop 2} \smallskip 

\ding{72} \textbf{Essay 1 Final Draft} \smallskip 

\ding{72} Reflective Journal \\ \hline

\end{tabular}
\end{small}



\begin{small}
\begin{tabular}{ | p{.6cm} | p{5cm} | p{6.5cm} | p{3.5cm}|}
\hline

10.14& 
& \emph{Peer Workshop}. &  \\ \hline

10.16& & \emph{Peer Workshop}. \smallskip

\ding{72} Bring two hard copies of your essay to class. & \ding{72} \textbf{Essay 2.0} \\ \hline

10.18& &  & 

\ding{72} Reflective Journal \\ \hline

\end{tabular}
\end{small}


\begin{small}
\begin{tabular}{ | p{.6cm} | p{5cm} | p{6.5cm} | p{3.5cm}|}
\hline

10.21& Wes Anderson, \emph{Rushmore}, 1998 (film). 
\smallskip

\href{http://libcat.dartmouth.edu/record=b2947849~S1}{\emph{Rushmore}} is on electronic reserve in Baker-Berry library's Jones Media Center. You should be able to view the film within \href{https://canvas.dartmouth.edu}{Canvas}. 
\smallskip

\href{https://github.com/stockphrase/OpenHandbook/raw/master/Open%20Handbook.pdf}{\emph{Open Handbook}}: read the ``Theoretical Writing'' section in ch. 3, ``Types of College Writing.''
& 
& \href{https://canvas.dartmouth.edu}{\ding{72} \textbf{Post \& Notes}}\\ \hline

10.23& & 

\ding{72} \emph{Workshop 4}: \emph{Close reading; working with images and film}.  & \\ 
\hline

10.25 && \emph{Student-led film readings}. & 

\ding{72} Reflective Journal\\

\hline


\end{tabular}
\end{small}



\begin{small}
\begin{tabular}{ | p{.6cm} | p{5cm} | p{6.5cm} | p{3.5cm}|}
\hline

10.28 & Matthew Crawford, ``The Case for Working with Your Hands.''

& 
& \href{https://canvas.dartmouth.edu}{\ding{72} \textbf{Post \& Notes}}\\ \hline

10.30& Re-read Crawford. &   & \\ \hline

11.1&  &   & \ding{72} \textbf{Essay 2 Final Draft} \smallskip

\ding{72} Reflective Journal \\ \hline

\end{tabular}
\end{small}


\begin{small}
\begin{tabular}{ | p{.6cm} | p{5cm} | p{6.5cm} | p{3.5cm}|}
\hline

11.4& & \emph{Peer Workshop}. \smallskip 

\ding{72} Bring two hard copies of your essay to class.

& \ding{72} \textbf{Essay 3.0}  \smallskip \\ \hline
11.6& &\emph{Peer Workshop}.  & \\ \hline
11.8& &\emph{Peer Workshop}. &\\ \hline 

\end{tabular}
\end{small}


\begin{small}
\begin{tabular}{ | p{.6cm} | p{5cm} | p{6.5cm} | p{3.5cm}|}
\hline

11.11& & Introduction to academic research. &\\ \hline
11.13&  & \emph{Guided research day I}.&\\ \hline
11.15& & \emph{Guided research day II}.& \\ \hline 
\end{tabular}
\end{small}

\begin{small}
\begin{tabular}{ | p{.6cm} | p{5cm} | p{6.5cm} | p{3.5cm}|}
\hline
11.18& & ePortfolio and Winter research projects. &\ding{72} \textbf{All Work Complete}\\ \hline
& & &\\ \hline
& &&\\ \hline 
\end{tabular}
\end{small}

\end{document}
